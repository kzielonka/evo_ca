\subsection{Reprezentacja danych}
Reprezetancja za pomoca wektora binarnego gdzie jedynke interpretujemy jako kondydature oferty do zaakceptowania, a zero za brak oferty w zbiorze ofert zaakceptowanych.
Jeżeli oferta posiada jedynke w wektorze czyli jest kandydatem do zaakceptowania nie oznacza jeszcze, że zostanie dodana do zbioru ofert zaakceptowanych.

Dzięki takiej reprezentacji możemy zastosować algorytm PBIL do rozwiązania tego problemu.

\subsection{Funkcja celu}
Oferty oznaczone jedynka rozumiemy jako oferty zaakceptowany. Dopuszczamy jednak możiwość występownia dwóch ofert zaakceptowanych o nierozłoącznym zbiorze towarów, ale wtedy preferujemy tą o mniejszym indeksie.
