Przeprowadzone testy wyraźnie pokazują, że algorytm SGA jest znacznie lepszy od algorytmów PBIL i algorytmu losowego.
Zdarzało się, że algorytm losowy dawał prawie tak dobre wyniki jak SGA lub równe, ale w większości przypadków był od niego gorszy.
Algorytm PBIL okazał się być trochę lepszy od algorytmu losowego, ale zdarzały się sytuacje, że oba były równe, a nawet algorytm losowy był lepszy.
Zaóważyliśmy, że trudność danych testowych dla algorytmu losowego, zależała od stosunku towarów do ofert.
Algorytm SGA i PBIL okazały się być znacznie od niego lepsze gdy liczba ofert znacznie przewyższa liczbę towarów (czyli jest znacznie więcej kolidujących ze sobą ofert).


Wydaje nam się, że najtrudniejszym  elementem w tych algorytmach było dobranie odpowiedniej reprezentacji danych.
Reprezentacje które wybraliśmy mają tą wadę, że niektóre zbiory zaakceptowanych ofert mogą być nierównomiernie reprezentowane przez osobniki.
W efekcie algorytm ma problemy ze znalezieniem optymalnego zbioru ofert, który może być reprezentowany przez bardzo małą liczbę osobników.


Dodatkowo, funkcja celu w pewnych momentach staje się skokowa -- dla danych z dużą liczbą kolizji ofert tylko początkowy fragment permutacji może mieć znaczenie przy obliczaniu zysku z aukcji. Utrudniło to zdecydowanie działanie standardowych operatorów mutacji, które często nie wpływały nawet na wynik funkcji przystosowania.
