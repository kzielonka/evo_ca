Przeprowadzone testy wyraźnie pokazują, że algorytm SGA jest znacznie lepszy od algorytmów PBIL i algorytmu losowego.
Zdarzało się, że algorytm losowy dawał prawie tak dobre wyniki jak SGA lub równe ale w większości przypadków był od niego gorszy.
Algorytm PBIL okazał się być troche lepszy od algorutmy losowego, ale zdarzały się sytuacje, że oba były równe, a nawet algorytm losowy był lepszy.
Zaóważyliśmy, że trudność danych testowych dla algorytmu losowego, zależa od stosunku towarów do ofert.
Algorytm SGA i PBIL okazały się być znacznie od niego lepsze gdy liczba ofert znacznie przewyższa liczbę towarów.


Wydaje nam się, że najtrudniejszymelementm w tych algorytmach to dobranie odpowiendiej reprezentacji danych.
Reprezentacje które wybraliśmy mają tą wade, że niektóre zbiory zaakceptowanych ofert mogą być nierównomiernie reprezentowane przez osobników.
W efekcie algorytm ma problemy ze znalezieniem optymalnego zbioru ofert, który może być reprezentowany przez bardzo małą liczbe osobników.
