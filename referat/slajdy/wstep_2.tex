\subsection{Opis problemu}
W aukcjach kombinatorycznych (ang. combinatorial auction) przedmiotem handlu jest wiele towarów.
Uczestnicy mogą składać oferty na zbiory towarów i te oferty są niepodzielne, tzn. muszą być przyjęte w całości lub w całości odrzucone. Problem wyznaczania zbioru ofert przyjętych maksymalizujących
        przychód w takiej aukcji jest w ogólnym przypadku NP trudnym problemem kombinatorycznym.
    % http://www.academia.edu/395279/ALGORYTMY_PRZYBLIZONEGO_ROZWI_AZYWANIA_PROBLEMU_AUKCJI_KOMBINATORYCZNEJ

\subsection{Dane i rozwiązania}
Dane to liczba towarów $n$ i $m$ ofert, gdzie każda oferta to lista towarów i proponowana za nią cena.
Oferty te są niepodzielne, a każdy przedmiot może zostać kupiony tylko raz.


Rozwiązaniem nazywamy zbiór ofert, w którym żadne dwie oferty nie zawierają tego samego przedmiotu.

\subsection{Funkcja celu}
Funkcją celu jest suma wartości ze zbioru $A$ wybranych ofert (przy czym zbiór ten spełnia wymagania zadania i oferty są niesprzeczne).

\begin{equation}
    f(A) = \sum\limits_{a \in A} cena(a)
\end{equation}

\subsection{Reprezentacja danych}
Naturalną reprezentacją byłby zbiór identyfikatorów ofert (liczb naturalnych) które zostały wybrane (w szczególności oferty te powinny być niesprzeczne). Ta reprezentacja nie pozwala jednak na efektywne stosowanie operatorów genetycznych i nie nadawała się do zastosowania w algorytmie ewolucyjnym.

\newpage

Postanowiliśmy wypróbować dwa rodzaje kodowania rozwiązania:
\begin{itemize}
    \item wektor binarny,
    \item permutacja liczb naturalnych.
\end{itemize}

Oba rozwiązania wymagały dodatkowej modyfikacji funkcji celu która na bieżąco odrzucała sprzeczne oferty (w związku z tym obliczenie wartości funkcji przystosowania danego osobnika stało się bardziej złożone czasowo).

\subsection{Dane testowe}
    Do generowania danych testów korzystamy z generatora CATS. Jest on najbardziej popularnym narzędziem dla tego problemu i jak twierdzi autor generuje dane zbliżone dla realnych problemów tego typu.






