\subsection{Opis problemu}
W aukcjach kombinatorycznych (ang. combinatorial auction) przedmiotem handlu jest wiele towarów.
Uczestnicy mogą składać oferty na zbiory towarów i te oferty są niepodzielne, tzn. muszą być przyjęte w całości lub w całości odrzucone. Problem wyznaczania zbioru ofert przyjętych maksymalizujących
        przychód w takiej aukcji jest w ogólnym przypadku NP trudnym problemem kombinatorycznym.
    % http://www.academia.edu/395279/ALGORYTMY_PRZYBLIZONEGO_ROZWI_AZYWANIA_PROBLEMU_AUKCJI_KOMBINATORYCZNEJ

\subsection{Dane i rozwiązania}
Dane to liczba towarów $n$ i $m$ ofert, gdzie każda oferta to lista towarów i proponowana za nią cena.
Oferty te są niepodzielne, a każdy przedmiot może zostać kupiony tylko raz.


Rozwiązaniem nazywamy zbiór ofert, w którym żadne dwie oferty nie zawierają tego samego przedmiotu.

\subsection{Funkcja celu}
Funkcją celu jest suma wartości ze zbioru $A$ wybrancyh ofert (przy czym zbiór ten spełnia wymagania zadania i oferty są niesprzeczne).

\begin{equation}
    f(A) = \sum\limits_{a \in A} cena(a)
\end{equation}

\subsection{Reprezentacja danych}
Uwagi:
\begin{itemize}
    \item naturalną reprezentacją byłby zbiór identyfikatorów ofert,
    \item ta reprezentacja nie pozwala jednak na efektywne stosowanie operatorów genetycznych
\end{itemize}
\vspace{1em}
Alternatywy:
\begin{itemize}
    \item wektor binarny
    \item permutacja
\end{itemize}

\subsection{Dane testowe}
    Do generowania danych testów korzystamy z generatora CATS. Jest on najbardziej popularnym narzędziem dla tego problemu i jak twierdzi autor generuje dane zbliżone dla realnych problemów tego typu.






