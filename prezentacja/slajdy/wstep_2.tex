\begin{frame}{Opis problemu}
    \begin{block}{Aukcje kombinatoryczne}
        W aukcjach kombinatorycznych (ang. combinatorial auction) przedmiotem handlu jest wiele towarów.
        Uczestnicy mogą składać oferty na zbiory towarów i te oferty są niepodzielne, tzn. muszą być przyjęte
        w całości lub w całości odrzucone. Problem wyznaczania zbioru ofert przyjętych maksymalizujących
        przychód w takiej aukcji jest w ogólnym przypadku NP trudnym problemem kombinatorycznym.
    \end{block}
    % http://www.academia.edu/395279/ALGORYTMY_PRZYBLIZONEGO_ROZWI_AZYWANIA_PROBLEMU_AUKCJI_KOMBINATORYCZNEJ
\end{frame}

\begin{frame}{Dane i rozwiązania}

    \begin{block}{Dane}
        Dane to liczba towarów $n$ i $m$ ofert, gdzie każda oferta to lista towarów i proponowana za nią cena.
        Oferty te są niepodzielne, a każdy przedmiot może zostać kupiony tylko raz.
    \end{block}

    \begin{block}{Rozwiązanie}
        Rozwiązaniem nazywamy zbiór ofert, w którym żadne dwie oferty nie zawierają tego samego przedmiotu.
    \end{block}

    \begin{block}{Funkcja celu}
        Funkcją celu jest suma wartości wybranych ofert (przy czym zbiór ten spełnia wymagania zadania i oferty są
        niesprzeczne).
    \end{block}

\end{frame}

\begin{frame}{Reprezentacja rozwiązania}
Uwagi:
\begin{itemize}
\item naturalną reprezentacją byłby zbiór identyfikatorów ofert,
\item ta reprezentacja nie pozwala jednak na efektywne stosowanie operatorów genetycznych
\end{itemize}
\vspace{1em}
Alternatywy:
\begin{itemize}
\item wektor binarny
\item permutacja
\end{itemize}
\end{frame}

\begin{frame}{Reprezentacja rozwiązania - wektor binarny}
Wektor binarny
\begin{itemize}
\item ciąg 0-1 długości równej liczbie ofert
\item 1 oznacza, że oferta została zaakceptowana, a 0, że została odrzucona
\item pozwala wykorzystać algorytm PBIL (wektor prawdopodobieństw)
\end{itemize}
\vspace{1em}
Problemy
\begin{itemize}
\item nie uwzględnia problemu niesprzeczności ofert,
\item ani nie pozwala na ustalenie kolejności ofert
\end{itemize}
Częściowo można to rozwiązać wybierając tylko niesprzeczne oferty zaczynając od lewej strony.
\end{frame}

\begin{frame}{Reprezentacja rozwiązania - permutacja}
Permutacja
\begin{itemize}
\item ciąg liczb naturalnych od 1 do n (liczba ofert)
\item permutacja oznacza kolejność rozpatrywania ofert
\item pozwala wykorzystać standardowy algorytm SGA
\end{itemize}
\vspace{1em}
Problemy
\begin{itemize}
\item niesprzeczność ofert jest rozwiązana akceptując niesprzeczne oferty zaczynając od lewej strony
\item w zależności od instancji problemu, wpływ na funkcję celu może mieć tylko niewielki początkowy fragment permutacji
\item wiele różnych osobników może mieć taki sam wynik funkcji celu
\end{itemize}
Wymienione problemy utrudniają generowanie osobników istotnie różnych (zarówno jako permutacja jak i wartość funkcji celu)
\end{frame}

\begin{frame}{Rozwiązanie problemu}
    \begin{itemize}
        \item Do rozwiązania problemu wykorzystujemy algorytm SGA.
        \item Operator krzyżowania to lekko zmodyfikowany operator PMX (staramy sie aby wymieniane środkowe segmenty były częściej wybierane z lewej storny osobnika niz prawej).
        \item Operator mutacji przesuwa całą permutacje o jeden element w lewo.
    \end{itemize}
\end{frame}

\begin{frame}{Dane testowe}
    Do generowania danych testów korzystamy z generatora CATS. Jest on najbardziej popularnym narzędziem dla tego problemu i jak twierdzi autor generuje dane zbliżone dla realnych problemów tego typu.
\end{frame}


