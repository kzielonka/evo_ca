\begin{frame}{Opis problemu}
    \begin{block}{Aukcje kombinatoryczne}
        W aukcjach kombinatorycznych (ang.combinatorialauc-tion) przedmiotem handlu jest wiele towarów. Uczestnicy mogą składać oferty na zbiory towarów i te oferty s ą nie-podzielne, tzn. muszą być przyjęte w całości lub w całości odrzucone. Problem wyznaczania zbioru ofert przyjętych maksymalizujących przychódw takiej aukcji jest w ogólnym przypadku NP -trudnym problemem kombinatorycznym.
    \end{block}
    % http://www.academia.edu/395279/ALGORYTMY_PRZYBLIZONEGO_ROZWI_AZYWANIA_PROBLEMU_AUKCJI_KOMBINATORYCZNEJ
\end{frame}

\begin{frame}{Representacja danych i rozwiązania}

    \begin{block}{Reprezentacja danych}
        Dane to liczba towarów $n$ i $m$ ofert, gdzie każda oferta to lista towarów i proponowana za nie cena.
    \end{block}

    \begin{block}{Reprezentajca rozwiązania}
        Permutacja ofert, którą rozumiemy jako kolejność rozpatrywania ofert.
        Oferta jest przyjmowana jeśli żaden z jej towarów nie został już wykupiony przez jakąs wcześniejszą ofertę.
    \end{block}

    \begin{block}{Funkcja celu}
        Akceptujemy oferty w kolejności występowania ich identyfikatorów w permutacji i wybierajać te, których towary są jeszcze dostępne.
        Jako wartość celu przyjmujemy sume cen zaakceptowanych ofert.
        %\begin{equation}
        %    f(x) = \sum\limits_{i=1}^{m} o_{m,1}
        %\end{equation}
    \end{block}
\end{frame}

\begin{frame}{Rozwiązanie problemu}
    \begin{itemize}
        \item Do rozwiązania problemu wykorzystujemy algorytm SGA.
        \item Operator krzyżowania to lekko zmodyfikowany operaotr PMX (staramy sie aby wymieniane środkowe segenty były częściej wybierane z lewej storny osobnika niz prawej).
        \item Operator mutacji przesuwa całą permutacje o jeden element w lewo.
    \end{itemize}
\end{frame}

\begin{frame}{Dane testowe}
    Do generowania danych testów korzystamy z generatora CATS. Jest on najbardziej popularnym narzedziem dla tego problemu i jak twierdzi autor generuje dane zbliżone dla realnych problemów tego typu.
\end{frame}


